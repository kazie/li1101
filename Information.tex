\documentclass[a4paper,11pt,exjobb]{kth-mag}

\usepackage[T1]{fontenc}
\usepackage{textcomp}
\usepackage{lmodern}
%\usepackage[swedish,english]{babel}
\usepackage[swedish]{babel} % This even does magic for bibtex! :D
\usepackage{modifications}

%\usepackage[natbib=true,style=alphabetic]{biblatex}
%\usepackage[natbib=true]{biblatex}
\usepackage[natbib=true,style=ieee]{biblatex}
\bibliography{Information}

%\usepackage{hyperref} % Throws a warning when included, but compiles just fine
\usepackage{multicol}
\usepackage{verbatim}
\usepackage{amsmath}
\usepackage{amsfonts}
\usepackage{xfrac}
\usepackage{hyperref}
%\usepackage{fancyref}

%\setcounter{tocdepth}{1} 
%\setcounter{secnumdepth}{1}
\def\thetitle{Är kryptovalutor* någonting för framtiden?}
\title{\thetitle}
\subtitle{*kryptovalutor är fullt digitala valutor som förlitar sig på kryptografi för att skapas och administreras. Största och mest kända heter Bitcoin (\textbaht, BTC, XBT).}
%\foreigntitle{Geografisk taktikröstning i svenska riksdagsval}

\author{Petter Salminen}
\date{\today}
\blurb{School of Computer Science, CSC\\Information Search LI1101, HT2013}
\trita{}

\begin{document}

\frontmatter
\maketitle

%\pagestyle{plain}
%\selectlanguage{english}
%\begin{abstract}

%\input{parts/abstract.en.tex}
%\end{abstract}

\clearpage
%\begin{foreignabstract}{swedish}
%\input{parts/abstract.sv.tex}
%\end{foreignabstract}

%\input{parts/foreword.tex}
%\vfill
%\input{parts/statement_of_collaboration.tex}
%\vfill
%\input{parts/document_change_record.tex}

\clearpage
\tableofcontents
\newpage

\mainmatter
\pagestyle{plain}
\chapter{Sökuppgiften}
%\setcounter{tocdepth}{1} 
%\setcounter{secnumdepth}{3}
\section{Notation}

Söksträngar är skrivna i verbatim, som exempel \verb="en sök sträng"=. Booleska operatorer är skrivna in \textsc{versal stil}, dessa är, \textsc{and}, \textsc{or} samt \textsc{not}.

\section{Databaser}
I uppgiften var det givet att man skulle använda 2 olika databaser för sin sökning. Men jag har tagit det aktiva valet att expandera detta till följande fem:
\textit{ACM Digital Library}\cite{ACM}, \textit{ArXiv.org}\cite{arxiv}, \textit{IEEE Xplore}\cite{IEEE}, \textit{Inspec}\cite{inspec} och \textit{Scopus}\cite{scopus}.

Förklarning till varför det blev fem, och inte endast två databaser följer i \ref{sec:whydata}.

\section{Sökningen}
Min initiala sökning som beskrivs i \ref{sec:first}, med analys av sökningen och resultat. Denna sökning förbättras till \ref{sec:second} genom att använda ord i \textit{Inspec} ur deras kontrollerade och kontrollerade ordlista.

Senare gick jag över till att eftersom min sökning handlar om något ganska nytt, och svårplacerat i deras tesaurus, så bestämde jag mig som beskrivs i \ref{sec:hard} att bredda min sökning med färre söktermer, men manuellt kolla på sammanfattningarna och titlarna för att göra en relevansvärdering.

Eftersom jag inte var ännu nöjd med mitt resultat, breddade jag även ut mig för att söka i fler databaser, detta beskrivs vidare i \ref{sec:database}.
\newpage
\subsection{Första försöket}
\label{sec:first}
Jag började att söka igenom databaserna \textit{IEEE Xplore} och \textit{Inspec} med följande fråga.
\begin{center}
\verb#("Digital currency")# \textsc{and} \verb#("future")#
\end{center}
Båda databaser gav under vanliga förutsättningar inga svar, däremot gav \textit{IEEE Xplore} 49st svar om man ändrade inställningarna på sökningen till en fulltext sökning.

\subsubsection{Analys av sökstrategi}
Själva sökningen gav inte några större resultat, vilket antagligen berodde på att orden ej var standardiserade och att jag endast fick resultat då jag sökte via fulltext sökning. Detta för att folk har i sina rapporter använt orden ``future'' och ``digital currency'' men dessa är vare sig 'taggar' eller välanvända ord i sammanfattningarna.

\subsubsection{Resultat}
Trotts detta fick jag en bra referens\cite{first}, som även hjälpte mig med att hitta bättre sökord inför nästa iteration. 
Den gav mig detta resultat i \textit{IEEE Xplore}, vilket antagligen skulle hjälpa mig med framtida sökningar. 
\begin{verbatim}
    INSPEC: CONTROLLED INDEXING
    electronic money.

    INSPEC: UNCONTROLLED INDEXING
    Bitcoin

    IEEE TERMS
    Cryptography
    Currency
\end{verbatim}


För övrigt hade  denna artikel en väldigt ögonfallande 'abstract', som inleddes med ``\textit{There's nothing like a dollar bill for paying a stripper.}''. Resten av sammanfattningen är bifogad i Bilaga \ref{app:abstract} och är hämtad från\cite{first}.

\subsection{Andra iterationen}
\label{sec:second}
Till andra sökningen har jag nu anpassat mig lite, och använder många fler ord än ``digital currency'' för att leta efter digitala valutor. Jag använder mig av \textsc{or} för att slå ihop sådant att en artikel kan innehålla något av de följande pengar-termena \textit{Digital currency}, \textit{Bitcoin}, \textit{electronic money} eller bara \textit{currency}. Sen binder jag ihop med att artikeln även måste vara taggad med orden \textit{cryptography} samt \textit{future} med \textsc{and}-operatorn.
 Detta gav mig följande söksträng.
\begin{center}
(\verb#"Digital currency"# \textsc{OR} \verb#"Bitcoin"# \textsc{OR} \verb#"electronic money"# \textsc{OR} \verb#"currency"#) \textsc{AND} (\verb#"cryptography"#) \textsc{AND} (\verb#"future"#)
\end{center}

Denna gång fick jag betydligt mycket fler svar , men precisionen var ännu ganska låg, och till syntes även recallen. \textit{IEEE Xplore} gav mig 5 resultat utan, och 554 resultat med fulltext sökning. Samt \textit{Inspec} gav mig denna gång 23 resultat.

\subsubsection{Analys av sökstrategi}
Tillägget av fler ord än ``digital currency'' gav betydligt fler resultat, speciellt eftersom dessa ord finns i tesaurus och är relativt vanliga och bra ord. Sen även är många saker som vanligen handlar om icke digitala pengar (om de skulle vara 'taggade' med ``currency'') bortfiltrerade av satsen ``\textsc{AND} (\verb#"cryptography"#)''.

Utan denna sats skulle det förekomma betydligt fler resultat i min resultatlista speciellt irrelevanta sådana.

\subsubsection{Resultat}
Trots att jag denna gång fick fler resultat, så hittade jag bara en\cite{second} som kändes relevant. Detta är en relativt gammal men titeln var extremt relevant till min fråga. ``Electronic cash-technology will denationalise money'' - ``Elektronisk pengar-teknik kommer avnationalisera pengarna''. Den handlar inte direkt om Bitcoins, utan mer om socio-ekonomiska effekter av teknologisk framgång.

Resultatet gav även mig ett bra ord för framtida sökningar, istället för ``future'' ska jag använda ``technological forecasting'' för artiklar som försöker förutse framtiden genom teknologiska förändringar. Om bra artiklar finns kan de hända använda denna 'tag'.

\subsection{Tredje iterationen}
\label{sec:hard}
I denna sökning så tänkte jag att jag mer eller mindre, egentligen bara ville ha saker taggade av ``technological forcasting'' istället för ``future'' samt utbredda min Bitcoin sökning till olika derivat genom att asterisk som ett jokertecken i slutet av bitcoin för att matcha med alla andra, exempelvis ``Bitcoins'', ``Bitcoin technology'' som jag har sett dyka upp i olika tesaurus. Jag kände att jag ville försöka vara mer precis, och därmed ta bort ``cryptography'' och ``currency''.
Detta gav mig följande sökning.
\begin{center}
(\verb#"bitcoin*"# \textsc{OR} \verb#"electronic money"#) \textsc{AND} (\verb#"technological forecasting"#)
\end{center}

Denna gång gav gick jag 2 resultat (12st via fulltext sökning) på \textit{IEEE Xplore} och 11 resultat via \textit{Inspec}.

\subsubsection{Analys av sökstrategi}
Min känsla här är att det inte finns speciellt mycket skrivet om ``Bitcoin'', och skälet till att jag får väldigt få resultat kan vara mitt mycket begränsade område, samt även att det är relativt nytt. Jag borde helt enkelt bara kolla hur många artiklar som innehåller ordet ``Bitcoin'' och om de är rätt få, gå igenom dem manuellt - samt, även använda andra databaser.

\subsubsection{Resultat}
Fastän jag fick väldigt få resultat så fick jag en artikel i ögonvrån\cite{tre}.
Denna måste läsas mer noggrant för att kunna analysera om detta val var en succé eller en flopp. Den handlar inte direkt om ``Bitcoin'' konstigt nog, utan mer generellt om virtuella pengar och inflation.

\subsection{Slutgiltiga sökningen}
\label{sec:database}
Jag bestämde mig sedan slutligen att bara söka efter ``Bitcoin*'' för att se hur många resultat detta skulle ge. 
Däremot fick jag inte några nya relevanta artiklar till min sökning, så jag bestämde mig att söka igenom några andra databaser efter ny information. Detta gav mig många bra papper om relevanta saker.

Jag sökte då även igenom databaserna \textit{ACM Digital Library}, \textit{ArXiv.org} och \textit{Scopus}\cite{scopus}.

\subsubsection{Analys av sökstrategi}
Just i denna uppgift verkar det som det är mer relevant att söka efter allt man kan hitta som innehåller ``Bitcoin'' snarare än att använda logik för att begränsa sökningen för att undvika att få för många träffar. Att jag bestämde mig att kolla igenom ännu fler databaser kändes som en bra idé och visar sig ha lönat sig något avsevärt.

\subsubsection{Resultat}
Sökningen på bara \verb#``bitcoin*''# gav mig inte många resultat. 34 (respektive 7) via \textit{IEEE Xplore} på vid fulltext läge, samt 21 resultat på \textit{Inspec}. 
Via \textit{Inspec} så hittade jag en artikel om pengartvättning då det gäller kryptovalutor\cite{Stokes}.
I \textit{ACM} fann jag 45 resultat, varav tre stycken intressanta\cite{Christ, Martins}.
\textit{ArXiV} fann jag 11 olika resultat, varav ett stort papper som påminner om ett exjobb om bitcoins\cite{bitsc}. 
\textit{Scopus} fann jag 26 resultat, varav tre stycken intressanta\cite{scoop1, scoop2, scoop3}.

\section{Diskussion om databaserna}
\label{sec:whydata}
De två första databaserna jag valde var \textit{IEEE Xplore} och \textit{Inspec} då dessa verkar vara något av de främsta artikel databaserna för data- och elektro-ingenjörer. Däremot kan man märka att man får en del krockar mellan dessa databaser då det mesta antas komma från ``Institution of Engineering and Technology''. Båda dessas sökverktyg fungerade ganska bra, dessvärre tror jag att det är mitt informationsbehov som sätter pinnen i hjulet för att visa hur pass bra de fungerar egentligen.

Detta gjorde att jag mot slutet lade till \textit{The ACM Digital Library} som har sedan länge varit en stor motståndare till IEEE dominans. Tyvärr gav dig mig endast en artikel som är intressant, men tyvärr inte nödvändigtvis den bästa till mitt informationsbehov. 

Då jag ändå inte var nöjd så använde jag även \textit{ArXiv.org} från Cornell Universitetet och fick en som jag tror är riktigt bra. \textit{ArXiv} har däremot ett lite sämre tillgängligt sökverktyg. Eftersom det är en liten ruta längst upp i hörnet förstår man inte att man kan göra vanliga sökningar i den.

Sist men inte minst använde jag även \textit{Scopus} för att få några extra artiklar. 

Alla databaserna verkade ha bristande information av det just jag behövde, men detta beror nog på att informationsmängden om ``Bitcoin'' är ganska låg. Utifrån mitt eget perspektiv så tror jag att den absolut bästa artikeln kan vara den jag fick från \textit{ArXiv}, som har ett långt papper med massor av diskussion, men även gott om referenser till tidigare och relevanta verk.

\section{Mitt informationsbehov}
Hur stort är mitt vidare informationsbehov efter dessa sökningar i alla dessa fem olika databaser?

Jag skulle själv påstå att mitt informationsbehov med den information jag nu mottagit är relativt låg. Jag har många bra artiklar däremot som jag måste läsa igenom för att kunna fullfölja min uppgift och besvara på frågan. 

Hur får jag tag på all denna information då? Ja, stora delar finns fulltexter länkade till via databaserna. Det var lite bökigt att göra det hemifrån, men jag kollade och såg att jag kunde ladda ned alla fulltexter via \textit{KTHB} för samtliga referenser. Skulle nu inte detta varit fallet, skulle jag behöva skicka en förfrågan till biblioteket sådant att dem skulle kunna fixa fram en kopia till mig.

Finns det kanske mer information jag inte hittat?
Antagligen finns detta, och jag antar att dessa kan finnas i andra databaser, precis som att jag hittade många fler då jag sökte i fler databaser, så tvivlar jag inte på att det finns mer i andra. Sen finns det nog inom andra mer socio-ekonomiska artiklar skrivna av ekonomer och samhällsvetare som kan ha funderat kring framtida pengarsystem, mindre direkt länkade till ``Bitcoin'' men principiellt samma sak - globala valutor med mera. 

\subsection{Val av litteratur}
Jag valde min litteratur först genom att kolla på titeln på dokumentet. Om det låter relevant eller intressant så gick jag vidare och läste sammanfattningen, ``abstract'', på engelska. Då jag hade gjort detta så fick jag i några få fall även skumma lite snabbt igenom fulltexten för att se om den är överblickande av intresse för sökfrågan.

\chapter{Litteraturstudie}
\section{Sammanfattning}
Digitala valutor har blivit allt mer centrala i den moderna människans liv. Men de traditionella digitala pengarna\footnote{Som ditt saldo du använder då du betalar med kort, vare sig internet eller i butik.} är inte lika anonymt som användningen av sedlar och mynt är. Bankerna kan till stor utsträckning kartlägga dina köp över var och när du har handlat, till viss mån även vad du har handlat. Detta har gett upphov till expansionen av kryptovalutor av olika sorter, varav den största och mest kända kallas Bitcoin. Men finns det någonting som talar för att digitala kryptovalutorna är här för att stanna?

Denna litteraturstudie av fyra artiklar, visar dock att frågeställningen tyvärr inte är en enkel Ja eller Nej fråga. Det är otroligt svårt att över huvud taget förutsäga hur teknologin helt kommer att arta sig i framtiden. Men helt ovissa blir vi inte av det, utan inser att det är en komplex fråga med den aktuella lutningen mot Ja-hållet.

\section{Inledning}
De senaste åren har det varit mycket hett ämnesområde att prata om internet och dess inverkan på ens privatliv. Sitter kanske egentligen storebror och lyssnar på allt du gör, endast i syftet att kartlägga dig och dina sysslor? Under 2012 har det varit stora nyheter om hur Edward Snowden har läckt ut hemlig information om hur USA och England har etablerat gigantiska program för just massövervakning. 

Detta har gjort att många, som till syntes kan ses som paranoida har sökt sig till anonymiseringsverktyg på internet för att försöka dölja sig från vakande ögon. Något man inte kan däremot göra, med landets och bankernas monopol på valutor och valutahandel är göra köp, fastän man skulle vilja vara anonym på internet. Kontanter kan du alltid överföra mellan personer utan att det skulle kännas spårningsbart, men detta gäller inte de traditionella digitala pengarna.

En lösning till detta problem till detta är kryptovalutor, digitala valutor som ej är kopplade till något land eller centrala valuta unioner. Den mest prominenta och populära kryptovalutan heter \textbf{Bitcoin}. Det speciella med kryptovalutor, som Bitcoin, är att de är denationaliserade, decentraliserade\footnote{Ingen äger eller styr valutan.} och mängden pengar som finns styrs strikt av en algoritm och matematik. Istället för traditionellt att ett land bestämmer mängden pengar som ska finnas i cirkulation.

Sen vill jag även anmärka att användningen av Bitcoin tyvärr kan användas till brottsliga ändamål. Att använda sig av anonymiseringsverktyg på internet och kombinera detta med att man kan göra transaktioner anonymt leder till att man kan sätta upp en svart marknad på internet, en av dessa kallas för \textit{Silk Road}\cite{Christ} där alla betalningar sker via Bitcoin.

Som student vid Datateknik - med inriktning mot datasäkerhet känner jag mig otroligt intresserad av allt detta som händer, speciellt då det kan vara relevant till vad jag ska göra exjobb inom, samt hur mitt framtida arbetsliv kanske myntas. 

\section{Bakgrund}
Med bakgrunden med tiden vi lever med, mitt intresse för ny teknik, datasäkerhet och kryptografi ställde jag mig frågan:
\begin{center}
\textit{Är kryptovalutor någonting för framtiden?}
\end{center}
Denna informationshunger hade jag tänkt mig släcka genom att gräva mig in i intressanta artiklar som jag har hittat inom det specifika ämnesområdet - mestadels relaterade till Bitcoin.

\section{Resultat}
Nedan presenteras en kort sammanfattning för var och en av de fem artiklar som studien baserar sig på.

\subsection{Artikel I}
% Cryptoanarchy
I \textit{IEEE Spectrum} hittar vi den iögonfallande artikeln \textit{``The Cryptoanarchists' Answer to Cash - How Bitcoin brought privacy to electronic transactions''}\cite{first} av Morgen E. Peck, som fint är myntat med texten \textsc{the future of money}, beskrivs hela bakgrunden och konceptet med Bitcoin för läsaren. Precis som jag har svårt att förutse framtiden för kryptovalutor, så har även denna inget definitivt svar. Det kan hända att det finns kvar i framtiden, men Bitcoin är bara första steget, och kommer antagligen själv utvecklas eller nya - bättre system kommer att komma på vägen. Detta kan enklast ses myntas av sista stycket i artikeln:
\begin{quote}
\textit{``If Bitcoin does fail, it may die in an
act of cannibalism. Nakamoto introduced the block chain, but cryptographers
are now already working on improvements. The minting rate is
only one of many things that could be tweaked. "Bitcoin is the first of
a new breed," says Garzik. "People will learn from Bitcoin and build
something better, or Bitcoin's critical mass will force it to evolve and
learn from its own mistakes."''}\cite[p.56]{first}.
\end{quote}

\subsection{Artikel II}
% Förbättringar till bitcoin
Nästa artikel, från Berkeley Universitetet i Kalifornien ifrågasätter om just Bitcoin kan överleva i det långa loppet, och undersöker dess svagheter och styrkor.  \textit{``Bitter to Better -- How to Make Bitcoin a Better Currency''}\cite{scoop3} frågar redan i sammanfattningen frågar \textit{``We ask also how Bitcoin could become a good candidate for a long-lived stable currency.''}. Men största delen av artikeln går ej ut på att svara på denna fråga, utan att prata först om \textit{Vad} Bitcoin är för att sen fortsätta med att beskriva dess svagheter och områden som kan förbättras.

Denna artikel är starkt kopplad till att själva Bitcoin kanske inte är för framtiden eftersom den är långt från perfekt \textit{``...Bitcoin is by no means perfect and some well-known problems are discussed later on...''}\cite[p.400]{scoop3}. Men kan med ändringar föreslagna i artikeln drar de slutsatsen att Bitcoin kan bli en pengaform för framtiden:

\begin{quote}
\textit{``We have provided a preliminary but broad study of the crypto-monetary phenomenon
Bitcoin, whose popularity has far overtaken the e-cash systems based on decades of
research. Bitcoin’s appeal lies in its simplicity, flexibility, and decentralization, making
it easy to grasp but hard to subvert. We studied this curious contraption with a critical
eye, trying to gauge its strengths and expose its flaws, suggesting solutions and research
directions. Our conclusion is nuanced: while the instantiation is impaired by its poor
parameters, the core design could support a robust decentralized currency if done right.''}\cite[p.414]{scoop3}.
\end{quote}

Bitcoin har själv några problem, och som tidigare artikeln så kan vi hittills se att den generella synen är att vi borde se till att fixa några av problemen, innan vi kan ansätta Bitcoin som något mer än en experimentell kryptovaluta.

\subsection{Artikel III}
% Det stora pappret
Tredje, är det absolut största pappret - vilket gav den från mig det fantasilösa namnet ``det-stora-pappret''.  \textit{``Bitcoin and Beyond: Exclusively Informational Money''}\cite{bitsc} går inte bara igenom vad Bitcoin är formellt, utan pengar i allmänhet och en extrem del läsning som är irrelevant för min fråga. Vid skrivandet av denna, så uppskattade författarna att chansen för Bitcoin lyckas, är väldigt låg - 1 på 100'000\cite[p.16]{bitsc}. Däremot skulle jag vilja ansätta att det inte direkt finns något vetenskapligt bakom dessa siffror och jag väljer att förkasta dem som sådant. 

Så detta papper är mindre intressant än de andra, men fortfarande följer trenden om att just Bitcoin antagligen kommer bli utbytt i framtiden till något annat. Men att det antagligen inte är slut på kryptovalutor om nu Bitcoin skulle dö ut.

\subsection{Artikel IV}
% Anthropology
\label{IV}
Den fjärde urskiljer sig från de andra tre på det sättet att den faktiskt inte är skriven av blivande Ingenjörer inom exempelvis Datalogi, utan den är skriven av medlemmar från Skolan för Antropologi i Irvine, Kalifornien.

\textit{``When perhaps the real problem is money itself!'': the practical materiality of Bitcoin}\cite{scoop1} som undersöker semiotiken, alltså hur man kan förknippa denna virtuella valuta till något verkligt. Att kunna materialisera  konceptet om kryptovalutor kan ses som något mycket viktigt för att kunna få folk att vilja acceptera Bitcoin och andra kryptovalutor. 

Ett enligt mig extremt värdefullt stycke ur texten finner vi här:
\begin{quote}
\textit{``Bitcoin is meaningful and
valuable not so much as an actual complementary or alternative currency, but
instead as an index of much broader discussions over the nature of money, credit and
capital in the world today. The monetary value of Bitcoin rests as much in the future
potential that its users imagine for it as on its current, relatively limited capacity to
act as a medium of exchange. Similarly, its semiotic value grows out of the
aspirations of Bitcoin adherents. The point is not whether Bitcoin ‘‘works’’ as a
currency, but what it promises: solidity, materiality, stability, anonymity, and,
strangely, community.''}\cite[p.263]{scoop1}
\end{quote}

Texten generellt är väldigt positiv till framtida användningen av kryptovalutor. Den ser helt enkelt ljust på att det kommer antagligen i alla fall finnas liknande valutor i framtiden.

\subsection{Artikel V}
% Silkroad
\label{V}
Den femte handlar mindre direkt om Bitcoin, utan om en anonym marknad på internet vid namnet \textit{Silk Road}. \textit{``Traveling the silk road: a measurement analysis of a large anonymous online marketplace''}\cite{Christ} är en artikel som försöker estimera hur stor omsättning som försegår på denna anonyma marknad på internet. Många har beskrivit \textit{Silk Road} som en svart marknad på internet, det råder hög anonymitet på denna sida genom användning av anonymitetstjänsten TOR\footnote{The Onion Router - \url{https://en.wikipedia.org/wiki/Tor_\%28anonymity_network\%29}}. Graden av anonymitet gör det möjligt att sälja olagliga produkter, vilket leder till att mestadel av handeln består av  droger.
\begin{quote}
\textit{``We were able to determine that Silk
Road indeed mostly caters drugs (although other items are also available), that it consists of a relatively international community,''}\cite[p.222]{Christ}.
\end{quote}

Ur denna artikel vill jag hämta ut två saker. Den första hur stor andel av handeln av Bitcoin försegår just på \textit{Silk Road}:
\begin{quote}
\textit{``The only conclusion we can draw from this comparison is that
Silk Road-related trades could plausibly correspond to 4.5\% to 9\%
of all exchange trades.''}\cite[p.220]{Christ}
\end{quote}

Samt att användningen av denna tjänst är ökande,  månadsvis finns en omsättning på över en miljon amerikanska dollar, vilket givetvis endast är handel som försegår med Bitcoin - en kryptovaluta. 
\begin{quote}
\textit{``We further discovered that the number of active
sellers and sales volume are increasing, corresponding, when averaged
over our measurement interval to slightly over USD 1.2 million per month for the entire marketplace...''}\cite[p.222]{Christ}
\end{quote}

\section{Diskussion}
Alla de fem utvalda litteraturena visar mer eller mindre stöd för att det kommer finnas någon sort av kryptovalutor i framtiden. Oavsett sig det kommer från det mer samhällsvetenskapliga hållet som\cite{scoop1}, tekniska universitet\cite{first,scoop3,bitsc,Christ} så menar man att det kommer finnas någon sort av kryptovalutor i framtiden. Detta beror på att det alltid kommer finnas folk som vill ha ett pengasystem som har egenskaperna av kryptovalutor. Att kunna göra anonyma transaktioner är väldigt värdefullt för en person som vill handla saker anonymt, som exempel tidigare i \nameref{V} beskrivs på en svart marknad \textit{Silk Road} som endast använder sig av kryptovalutan Bitcoin. Andra skäl till att det finns en kraft som vill ha dessa krypotovalutor är att de är denationaliserade. Detta gör att växlingskurser inte existerar - det är även decentraliserat vilket gör överföringar otroligt billiga. Som menat av skrivet av \nameref{IV}, så kan man se kryptovalutor som digitala ädelmetaller:

\begin{quote}
This practical materialism involves a potent discursive investment by Bitcoin
users in the determinist mechanics of Bitcoin’s code: first, its apparent ‘‘privatization’’ of the identities of those transacting Bitcoins (through network decentralization and cryptography); and second, its ‘‘digital metallism,’’ the supposed grounding of value, inspired by gold standard economics, through algorithmic control of the money supply.\cite[p.262]{scoop1}
\end{quote}  

Det är tyvärr väldigt svårt att göra tekniska förutsägelser. Däremot har jag inte har kunnat hitta något för att påpeka just motsatsen till att just kryptovalutor är en lösning på problemen med dagens elektroniska pengar. Om det finns folk som vill göra anonyma transaktioner av pengar på internet, så kommer det finnas någon sort av lösning till hands - denna kan vara även i framtiden kryptovalutor. Däremot tvivlar jag på att det kommer bli ett komplett paradigmskifte i finansvärlden, kryptovalutor kommer mest användas av intresserade  som ett komplement till vanliga pengar.

Avslutningsvis vill jag påpeka att det inte finns en stor valmöjlighet vad gäller litteratur som denna litteraturstudie bygger sig på. Det är väldigt lite skrivet om just kryptografiska valutor, Bitcoin skapades 2009 (och den stora starten på användningen av kryptovalutor därmed) så ämnet betraktas som ett relativt nytt. Det finns otvivelaktigt massor med resurser om hur olika kryptovalutor fungerar. Men det var ju inte det som var frågan som undersöktes, frågan var om kryptovalutor  var något för framtiden. Här måste man ta ett steg in i det okända eftersom detta handlar om någonting vi aldrig kommer kunna få svar om - framtiden. Men det är mycket sannolikt att kryptovalutor är den digitala och anonyma valutan som en viss del folk vill se och ha på internet.

\cleardoublepage
\printbibliography

%listoffigures
%listoftables

\cleardoublepage
\appendix
\pagestyle{empty}
\chapter{Appendix}
\section{The cryptoanarchists' answer to cash}
\subsection{Abstract}
\label{app:abstract}
There's nothing like a dollar bill for paying a stripper. 

Anonymous, yet highly personal-wherever you use it, that dollar will fit the occasion. Purveyors of Internet smut, after years of hiding charges on credit cards, or just giving it away for free, recently found their own version of the dollar-a new digital currency called Bitcoin.

\section{From Göran}
Got these from Göran\cite{goran1, goran2}.

Konferanser.

Båda två har inga fulltexter på internet. (Conference Paper)\cite{goran2}

%\addappheadtotoc

\end{document}
